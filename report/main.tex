%!TEX program = xelatex
\documentclass[UTF8,a4paper,11pt]{ctexart}

% ---------- 基础排版 ----------
\usepackage[a4paper,margin=2.5cm]{geometry}
\usepackage{setspace}
\setstretch{1.12}
\setlength{\parindent}{2em}
\setlength{\parskip}{0.35em}

% ---------- 数学与符号 ----------
\usepackage{amsmath,amssymb,bm}

% ---------- 表格 ----------
\usepackage{booktabs}
\usepackage{tabularx}
\usepackage{longtable}
\usepackage{array}
\usepackage{multirow}

% ---------- 图片 ----------
\usepackage{graphicx}
\usepackage{float}

% ---------- 超链接 ----------
\usepackage[hidelinks]{hyperref}

% ---------- 代码块 ----------
\usepackage{listings}
\usepackage{xcolor}
\lstset{
  basicstyle=\ttfamily\small,
  columns=fullflexible,
  breaklines=true,
  frame=single,
  numbers=left,
  numberstyle=\tiny,
  xleftmargin=2em,
  framexleftmargin=1.5em
}

% ---------- 摘要样式 ----------
\usepackage{abstract}

% 标题"摘要"的字体与位置
\renewcommand{\abstractname}{摘\quad 要}
\renewcommand{\abstractnamefont}{\bfseries\zihao{4}\centering}

% ---------- 彩色框 ----------
\usepackage{tcolorbox}

% 摘要正文:字号 + 行距(可调 1.25/1.30/1.35)
\renewcommand{\abstracttextfont}{\zihao{-4}\setstretch{1.32}}

% 摘要左右缩进:缩窄行宽→自动多换行→占用更“高”(可调 1.5cm~2.5cm)
\setlength{\absleftindent}{2.0cm}
\setlength{\absrightindent}{2.0cm}

% 摘要段首缩进(中文摘要一般保留段首缩进)
\setlength{\absparindent}{2em}

% 摘要上下留白(按审美微调)
\setlength{\abstitleskip}{0.6em}

% ---------- 小工具 ----------
\newcommand{\code}[1]{\texttt{\detokenize{#1}}}

% 定义 subsubsubsection(使用 paragraph 样式)
\newcommand{\subsubsubsection}[1]{\paragraph{#1}\mbox{}\\}

% 如果图片文件不存在,用占位框保证可编译
\newcommand{\includefigure}[3][]{%
\begin{figure}[H]
  \centering
  \IfFileExists{#2}{%
    \includegraphics[#1]{#2}%
  }{%
    \fbox{\parbox{0.85\linewidth}{\centering
      \small 图像文件缺失:\detokenize{#2}\\
      请在 Overleaf 上传该文件,或修改路径
    }}%
  }%
  \caption{#3}
\end{figure}
}

\begin{document}

% --- 1. 封面 ---
\begin{titlepage}
    \thispagestyle{empty}
    \centering

    \includegraphics[width=0.8\textwidth]{GSAI.png}
    
    \vspace{3cm}
    
    {\Huge \textbf{《数字信号处理》2025秋\\Group2实验报告}}

    \vfill
    
    {\large 
    \renewcommand{\arraystretch}{1.5}
    \begin{tabular}{rl}
        课程:& \underline{\makebox[6cm][c]{数字信号处理}}\\
        组长:& \underline{\makebox[6cm][c]{刘嘉俊}}\\
        组员:& \underline{\makebox[6cm][c]{孙浩翔、田原、叶栩言、林梓杰}}\\
        日期:& \underline{\makebox[6cm][c]{2025年12月20日}}\\
    \end{tabular}
    }
    \vspace{2cm}
    
\end{titlepage}

% --- 2. 目录 ---
\newpage
\pagenumbering{Roman} % 目录页使用罗马数字 (I, II, III)
\tableofcontents      % 生成目录
\newpage
\pagenumbering{arabic} % 正文重新开始使用阿拉伯数字 (1, 2, 3)

% --- 3. 摘要 ---

\begin{abstract}
本文以 ESC-50 环境声音数据集为基准,面向示例查询式(Query-by-Example, QbE)音频检索与环境声音分类两项核心任务,构建了从特征提取、表征学习到检索策略与训练配方的系统化评测框架。针对检索任务,报告覆盖传统信号处理方法、监督深度模型以及大规模预训练音频编码器等多类路线,并在统一实验协议下采用 Hit@K、MRR、mAP 与 NDCG 等指标进行对比分析。除总体性能外,进一步围绕 MFCC 关键超参数(帧长/帧移等)、预加重、Mel 标度选择、倒谱均值方差归一化(CMVN)策略等环节开展消融,澄清“工程细节”对检索效果的影响机制:例如,全局 CMVN 能稳定提升检索性能,而逐条 CMVN 与简单统计池化的组合可能引发表征退化。结果显示,预训练模型(以 CLAP 为代表)在检索任务上具有显著优势,能够在语义对齐的嵌入空间中实现近乎“开箱即用”的高召回;相比之下,DTW 等序列对齐方法虽然优于纯池化特征,但总体仍受限于表示能力与计算成本。

在效率与鲁棒性维度,报告从检索延迟、吞吐(QPS)与候选库内存占用三方面刻画精度—效率权衡,指出 BoAW 等方法可提供高吞吐的工程化方案,而 DTW 适用于离线或小规模场景。为同时兼顾准确率与速度,进一步引入融合与两阶段检索策略:利用轻量方法进行粗召回,再以 DTW 对候选集重排,可在较小候选池规模下实现显著加速,并在一定条件下维持甚至略增检索准确率。鲁棒性实验系统考察加性噪声、音量扰动、语速变化、变调与时间偏移等因素,表明加性噪声与语速变化是主要退化来源,而随机时间偏移影响相对有限。针对分类任务,报告以预训练 CLAP 特征上的线性探测(Linear Probing)为强基线,并通过测试时增强(TTA)、多尺度统计特征融合、深层分类器、标签平滑与模型集成等手段构建逐级优化路径,实现从“快速可用”到“高精度”的性能提升;同时在端到端微调对比中,验证 SpecAugment、Mixup 与延迟解冻等正则化策略对小数据集泛化的重要性,并展示基于声学标记器的 Transformer 预训练模型(BEATs)相较传统 CNN 架构在数据受限场景下的优势。综上,报告给出了在 ESC-50 场景下兼顾效果、效率与鲁棒性的可复用基准与实践建议,为后续环境声音检索与分类系统的工程落地与方法改进提供了依据。
\end{abstract}

\noindent\textbf{关键词:} 环境声音检索;ESC-50;CLAP;DTW;两阶段检索;环境声音分类;BEATs

\newpage

\part{任务一:声音检索}

% =========================================================
\section{引言}

\subsection{问题定义}

\textbf{示例查询式(QbE)音频检索} 旨在解决这样一个问题:给定一段查询音频,从数据库中检索语义上相似的录音。与基于文本的检索不同,QbE 直接对声学信号进行操作,使直观的“用声音搜声音”成为可能。形式化地,给定查询音频 $q$ 与候选库 $\mathcal{G}=\{g_1,g_2,\dots,g_N\}$,系统计算相似度 $s(q,g_i)$,并返回最相似的前 $K$ 个候选。
QbE在音效库、环境监测、音乐信息检索、生物声学等方面均有极大的应用价值。基于此背景,本文探究\textbf{QbE}的不同算法与超参组合在\textbf{ESC-50 数据集}上的表现。

\subsection{数据集:ESC-50}

我们在 \textbf{ESC-50 数据集}(环境声音分类)上进行评估,它包含 2,000 条环境声音录音,组织为 50 个语义类别(每类 40 条样本)。每条录音为 5 秒,采样率 44.1 kHz,本实验下采样到 22,050 Hz。数据集提供预定义的 5 折交叉验证划分,确保各折类别分布均衡。


\includefigure[width=0.8\linewidth]{image.png}{ESC-50 数据集示例}


% =========================================================
\section{方法}

\subsection{信号处理算法实现}

\subsubsection{预加重滤波器}

为补偿语音与环境声音的自然谱倾斜(低频占主导),我们可选地施加一阶高通预加重滤波器:
\[
y[n]=x[n]-\alpha\cdot x[n-1]
\]
其中 $\alpha=0.97$ 为预加重系数。该操作可以增强高频成分,提高判别性信息常出现的高频段信噪比,从而提升特征显著性。

\subsubsection{短时傅里叶变换(STFT)}

STFT 通过对重叠窗口片段施加离散傅里叶变换(DFT),将时域信号分解为时频表示:
\[
X(m,k)=\sum_{n=0}^{N-1} x[n+mH]\cdot w[n]\cdot e^{-j2\pi kn/N}
\]
其中:
\begin{itemize}
  \item $m$ 表示帧索引
  \item $k$ 表示频率 bin 索引($k=0,1,\dots,N/2$)
  \item $N$ 为 FFT 长度(默认:\code{n_fft = 2048},在 22,050 Hz 下约 93ms)
  \item $H$ 为帧移(默认:\code{hop_length = 512} 采样点,约 23ms)
  \item $w[n]$ 为分析窗函数
\end{itemize}

\textit{注意}:以上是 \textbf{librosa 的默认参数},并非领域标准值。传统语音处理通常使用更短的窗(约 25ms)与帧移(约 10ms)。我们的网格搜索实验发现最优性能在 40ms 帧长(\code{n_fft = 882})和 5ms 帧移(\code{hop_length = 110})处取得。

我们采用 \textbf{Hann 窗} 以降低谱泄漏:
\[
w[n]=0.5\left(1-\cos\frac{2\pi n}{N-1}\right)
\]

\textbf{功率谱}可由幅度平方得到:
\[
P(m,k)=|X(m,k)|^2
\]

\subsubsection{Mel 频谱}

人类听觉对频率的感知接近对数尺度。Mel 标度用于近似这种感知变换:

\textbf{HTK 公式:}
\[
m(f)=2595\cdot \log_{10}\left(1+\frac{f}{700}\right)\equiv 1127\cdot \ln\left(1+\frac{f}{700}\right)
\]
注:上式两种形式在数学上等价(只差对数底的换算:$2595/\log(10)\approx 1127$)。实践中,librosa 的 ``Slaney'' 实现采用 \textbf{分段线性-对数公式},在 1 kHz 以下为线性、以上为对数,更贴近低频的感知;``HTK'' 选项则在全频段使用纯对数公式。两者可通过 \code{htk} 参数切换。

我们构建 $K$ 个在 Mel 标度上均匀间隔的三角滤波器组成滤组。Mel 频谱通过将该滤组施加到功率谱上获得:
\[
M_k=\sum_{f=0}^{N/2}P(f)\cdot H_k(f)
\]
其中 $H_k(f)$ 表示第 $k$ 个三角滤波器在频率 bin $f$ 处的响应。

\textbf{对数压缩}近似人耳的对数响度感知:
\[
S_k=\log(M_k+\epsilon),\quad \epsilon=10^{-10}
\]
小常数 $\epsilon$ 用于避免近静音帧的数值不稳定。

\subsubsection{Mel 频率倒谱系数(MFCC)}

MFCC 通过 \textbf{离散余弦变换(DCT-II)} 对 log-Mel 频谱进行去相关:
\[
c_n=\alpha_n\sqrt{\frac{2}{K}}\sum_{k=0}^{K-1}S_k\cdot \cos\left(\frac{\pi n(k+0.5)}{K}\right)
\]
其中 $K$ 为 Mel 频带数(\code{n_mels = 128}),$n$ 为倒谱系数索引($n=0,1,\dots,\text{n\_mfcc}-1$),$\alpha_n$ 为正交归一化系数:
\[
\alpha_n=\begin{cases}
1/\sqrt{2} & \text{if } n=0\\
1 & \text{otherwise}
\end{cases}
\]
我们采用 \textbf{正交归一化}(\code{norm='ortho'}),以在变换中保持能量。

低阶 MFCC 描述谱包络(音色),高阶系数编码更细的频谱细节。池化类方法提取 \code{n_mfcc = 20},DTW 使用 \code{n_mfcc = 13} 以在判别性与计算成本间折中。

\subsubsection{倒谱升举(可选)}

升举通过正弦窗对低阶倒谱系数进行去强调:
\[
\hat{c}_n=c_n\cdot \left(1+\frac{L}{2}\sin\frac{\pi n}{L}\right)
\]
其中 $L$ 为升举参数(通常取 22),可提升对信道变化的鲁棒性。

\subsubsection{Delta 与 Delta-Delta 特征}

时间动态通过 \textbf{回归式导数} 捕获:
\[
\Delta c_t=\frac{\sum_{n=1}^{N}n(c_{t+n}-c_{t-n})}{2\sum_{n=1}^{N}n^2}
\]
其中 $N=(\text{width}-1)/2$,\code{width = 9} 帧。Delta-delta($\Delta\Delta$)通过对 delta 再做一次相同运算获得,反映加速度信息。静态 + delta + delta-delta 的拼接将特征维度增至 3 倍,但能更好地表征谱随时间的变化。

\subsubsection{倒谱均值与方差归一化(CMVN)}

CMVN 通过标准化去除信道效应:
\[
\hat{c}_{t,d}=\frac{c_{t,d}-\mu_d}{\sigma_d}
\]
其中 $\mu_d$ 与 $\sigma_d$ 分别为第 $d$ 个系数的均值与标准差。\textbf{全局 CMVN} 在整个数据集上计算统计量,而 \textbf{逐条语音级 CMVN} 对每条音频独立归一化。

\subsubsection{特征池化}

可变长度的帧序列通过 \textbf{时间池化} 转为定长嵌入:

\textbf{均值-标准差池化:}
\[
\mathbf{v}=[\mu_1,\dots,\mu_D,\sigma_1,\dots,\sigma_D]
\]
其中
\[
\mu_d=\frac{1}{T}\sum_{t=1}^{T}f_{t,d},\quad
\sigma_d=\sqrt{\frac{1}{T}\sum_{t=1}^{T}(f_{t,d}-\mu_d)^2}
\]
该向量维度为 $2D$,同时编码平均谱特征及其波动。

\subsubsection{自实现核心算法性能分析}

本项目要求自主实现 FFT、STFT、MFCC 等核心信号处理算法,不直接调用 SciPy/librosa 的核心函数。以下对各模块的实现方案与性能进行分析。

\textbf{(1)FFT 实现}

我们采用 \textbf{Cooley-Tukey Radix-2 DIT}(Decimation-In-Time)迭代式算法实现 FFT,复杂度为 $O(N\log N)$。为提升性能,使用 \textbf{Numba JIT} 编译为本地机器码,并支持并行批量处理。

\begin{table}[H]
\centering
\caption{FFT 精度与性能对比(vs SciPy)}
\begin{tabular}{@{}rrrrc@{}}
\toprule
长度 $N$ & 自实现 (ms) & SciPy (ms) & 速度比 & 最大误差 \\
\midrule
256  & 0.012 & 0.007 & 1.7$\times$ & $8.4\times10^{-14}$ \\
1024 & 0.029 & 0.013 & 2.3$\times$ & $1.3\times10^{-12}$ \\
4096 & 0.108 & 0.040 & 2.7$\times$ & $8.0\times10^{-12}$ \\
8192 & 0.226 & 0.081 & 2.8$\times$ & $1.0\times10^{-11}$ \\
\bottomrule
\end{tabular}
\end{table}

自实现 FFT 与 SciPy 的最大误差在 $10^{-11}$ 量级,满足双精度浮点数的理论精度要求。速度约为 SciPy 的 2--3 倍慢,这是合理的,因为 SciPy 底层使用高度优化的 C/Fortran 库(FFTW 或 Intel MKL)。

\textbf{(2)STFT 实现}

STFT 基于自实现的 FFT 构建,采用以下优化策略:
\begin{itemize}
    \item \textbf{向量化帧提取}:使用 NumPy 高级索引一次性提取所有帧
    \item \textbf{批量 FFT}:调用 Numba 并行的 \code{rfft_batch()} 函数处理所有帧
    \item \textbf{多种窗函数}:支持 Hann、Hamming、Blackman、Bartlett、Tukey 窗
\end{itemize}

\begin{table}[H]
\centering
\caption{STFT 精度验证(vs librosa)}
\begin{tabular}{@{}llc@{}}
\toprule
测试项 & 最大误差 & 状态 \\
\midrule
STFT (Hann 窗)     & $1.09\times10^{-12}$ & \checkmark \\
STFT (Hamming 窗)  & $1.10\times10^{-12}$ & \checkmark \\
STFT (Blackman 窗) & $1.05\times10^{-12}$ & \checkmark \\
ISTFT 重建         & $6.97\times10^{-14}$ & \checkmark \\
\bottomrule
\end{tabular}
\end{table}

STFT 输出与 librosa 完全一致(误差 $<10^{-12}$),ISTFT 可实现完美重建(误差 $<10^{-13}$)。

\textbf{(3)MFCC 实现}

MFCC 流水线完全基于自实现组件构建:

\begin{enumerate}
    \item \textbf{STFT}:调用自实现的 \code{stft()} 函数
    \item \textbf{功率谱}:$P(m,k) = |X(m,k)|^2$
    \item \textbf{Mel 滤波器组}:使用 Slaney 公式进行 Hz$\leftrightarrow$Mel 转换,构建三角滤波器
    \item \textbf{对数压缩}:$S_k = 10\log_{10}(M_k + \epsilon)$
    \item \textbf{DCT-II}:离散余弦变换,采用正交归一化
\end{enumerate}

\begin{table}[H]
\centering
\caption{MFCC 各组件精度验证(vs librosa/scipy)}
\begin{tabular}{@{}llc@{}}
\toprule
组件 & 最大误差 & 状态 \\
\midrule
Mel 滤波器组 & $2.58\times10^{-9}$  & \checkmark \\
DCT-II       & $3.53\times10^{-14}$ & \checkmark \\
完整 MFCC    & $2.46\times10^{-7}$  & \checkmark \\
Delta 特征   & $<10^{-10}$          & \checkmark \\
\bottomrule
\end{tabular}
\end{table}

\textbf{(4)性能总结}

\begin{table}[H]
\centering
\caption{自实现算法性能总结}
\begin{tabular}{@{}llll@{}}
\toprule
模块 & 核心优化技术 & 精度 & 相对标准库 \\
\midrule
FFT  & Numba JIT + 并行批处理 & $<10^{-11}$ & 2--3$\times$ 慢 \\
STFT & 向量化 + 批量 FFT      & $<10^{-12}$ & 可接受 \\
MFCC & 端到端自实现           & $<10^{-6}$  & 可接受 \\
\bottomrule
\end{tabular}
\end{table}

自实现代码在保证数值精度的前提下,性能处于合理范围。与 SciPy/librosa 的速度差距主要源于后者使用 C/Fortran 编写的高度优化库。在实际应用中,Numba JIT 的首次编译开销可通过缓存消除,批量处理场景下性能差距进一步缩小。

\subsection{检索方法}

我们评估 \textbf{13 种检索方法},分为三类。具体评估结果见第三部分。

\subsubsection{传统方法(M1-M7)}

\begin{table}[H]
\centering
\caption{传统方法(M1-M7)}
\begin{tabular}{@{}llll@{}}
\toprule
方法 & 特征 & 表示方式 & 距离度量 \\
\midrule
\textbf{M1}: MFCC+Pool+Cos & MFCC (20) & 均值+标准差池化 & 余弦 \\
\textbf{M2}: MFCC+Delta+Pool & MFCC+$\Delta$+$\Delta\Delta$ (60) & 均值+标准差池化 & 余弦 \\
\textbf{M3}: LogMel+Pool & Log-Mel (128) & 均值+标准差池化 & 余弦 \\
\textbf{M4}: Spectral+Stat & 频谱特征 (7) & 统计量拼接 & L2 \\
\textbf{M5}: MFCC+DTW & MFCC (13) & 帧序列 & DTW \\
\textbf{M6}: BoAW+ChiSq & MFCC (13) & 直方图 ($K=128$) & 卡方 \\
\textbf{M7}: MultiRes+Fusion & 短窗+长窗 & 加权融合 & 组合 \\
\bottomrule
\end{tabular}
\end{table}

\subsubsection{深度学习方法}

\begin{table}[H]
\centering
\caption{深度学习方法}
\begin{tabular}{@{}llll@{}}
\toprule
方法 & 架构 & 训练 & 代码位置 \\
\midrule
\textbf{CNN} & 5 个卷积块 & 监督训练(ESC-50) & \code{src/models/cnn_classifier.py:51-66} \\
\textbf{Autoencoder} & 卷积自编码器 & 无监督 & \code{src/models/autoencoder.py:29-77} \\
\textbf{Contrastive} & CNN + SupCon & 监督训练 & \code{src/models/contrastive.py:45-80} \\
\bottomrule
\end{tabular}
\end{table}

\subsubsection{预训练方法}

\begin{table}[H]
\centering
\caption{预训练方法}
\begin{tabular}{@{}llll@{}}
\toprule
方法 & 模型 & 预训练数据 & 嵌入维度 \\
\midrule
\textbf{M8}: CLAP & HTSAT-base & AudioSet + text & 512 \\
\textbf{M9}: Hybrid & CLAP + MFCC & 融合 & 512+40 \\
\textbf{BEATs} & 音频 Transformer & AudioSet & 768 \\
\bottomrule
\end{tabular}
\end{table}

\subsection{相似度度量}

\subsubsection{余弦距离}

对于池化后的嵌入,我们计算 \textbf{余弦距离}:
\[
d_{\cos}(\mathbf{q},\mathbf{g})
=1-\frac{\mathbf{q}\cdot \mathbf{g}}{\|\mathbf{q}\|_2\|\mathbf{g}\|_2}
\]
余弦距离与尺度无关,关注向量之间的夹角关系。

\subsubsection{动态时间规整(DTW)}

DTW 通过寻找最优扭曲路径来对齐两条时间序列,使累积距离最小。递推关系为:
\[
D(i,j)=d(x_i,y_j)+\min\{D(i-1,j),D(i,j-1),D(i-1,j-1)\}
\]
其中 $d(x_i,y_j)=\|x_i-y_j\|_2$ 为帧向量的欧氏距离。最终 DTW 距离为长度分别为 $I$ 与 $J$ 的序列在 $D(I,J)$ 处的值。

\textbf{Sakoe-Chiba 带状约束} 将扭曲路径限制在主对角线半径 $r$ 的范围内:$|i\cdot J/I-j|\le r$。在基线实验中我们使用无约束 DTW(\code{sakoe_chiba_radius = -1})。

\subsubsection{卡方距离}

用于 Bag-of-Audio-Words(BoAW)直方图表示:
\[
\chi^2(\mathbf{p},\mathbf{q})
=\sum_{i=1}^{K}\frac{(p_i-q_i)^2}{p_i+q_i+\epsilon}
\]
其中 $K=128$ 为码本大小,$\epsilon$ 用于防止除零。

\subsection{实现参数}

\begin{table}[H]
\centering
\caption{实现参数}
\begin{tabular}{@{}lll@{}}
\toprule
参数 & 值 & 说明 \\
\midrule
采样率 & 22,050 Hz & 音频重采样目标 \\
n\_fft & 2048 & FFT 窗长(约 93ms) \\
hop\_length & 512 & 帧移(约 23ms,75\% 重叠) \\
n\_mels & 128 & Mel 滤波器组大小 \\
n\_mfcc & 20(池化),13(DTW) & MFCC 系数数目 \\
窗函数 & Hann & 分析窗 \\
DCT 类型 & 2(正交归一) & DCT 变体 \\
Delta 宽度 & 9 帧 & 导数窗口 \\
BoAW 聚类数 & 128 & 码本大小 \\
预加重 & 0.97(可选) & 高通系数 \\
\bottomrule
\end{tabular}
\end{table}

\subsection{速度优化技术}

为提升检索效率,我们在实现中采用了多种优化技术。DTW 检索使用 \textbf{Numba JIT 编译}加速,将 Python 循环编译为高效的机器码,并结合 \code{prange} 实现多核并行计算,使 DTW 距离计算速度提升约 $10\times$。此外,我们还实现了 \textbf{Sakoe-Chiba 带状约束}将复杂度从 $O(N^2)$ 降至 $O(N \cdot R)$、\textbf{候选库特征缓存}避免重复计算、\textbf{两阶段检索}(粗召回 + 精排)在保持精度的同时实现 $2.68\times$ 加速、以及 \textbf{GPU 加速}的向量化距离计算。所有优化均已在代码中实现(详见 \code{src/retrieval/} 目录)。

% =========================================================
\section{实验设置}

\subsection{评测协议}

我们采用 ESC-50 的 \textbf{5 折交叉验证}。在每一折中:
\begin{itemize}
  \item \textbf{候选库}:1,600 条样本(fold 1-4)
  \item \textbf{查询集}:400 条样本(fold 5,轮换)
\end{itemize}

对每条查询,我们从候选库中检索前 $K$ 个候选,并判断其是否与查询具有相同类别标签。

\subsection{评测指标}

\begin{table}[H]
\centering
\caption{评测指标}
\begin{tabular}{@{}lll@{}}
\toprule
指标 & 公式 & 含义 \\
\midrule
\textbf{Hit@K} & $\mathbb{1}[\exists i\le K: y_i=y_q]$ & 二值:Top-K 内是否有正确项 \\
\textbf{Precision@K} & $\frac{1}{K}\sum_{i=1}^{K}\mathbb{1}[y_i=y_q]$ & Top-K 中正确比例 \\
\textbf{MRR@K} & $\frac{1}{\text{rank}_1}$ & 首个正确项的倒数排名 \\
\textbf{AP@K} & $\frac{1}{\min(K,R)}\sum_{k=1}^{K}P@k\cdot \mathbb{1}[y_k=y_q]$ & P-R 曲线下的面积 \\
\textbf{mAP@K} & 对所有查询的 AP@K 取平均 & 总体检索质量 \\
\textbf{NDCG@K} & $\frac{DCG@K}{IDCG@K}$ & 位置折损相关性 \\
\bottomrule
\end{tabular}
\end{table}

我们在 $K\in\{1,5,10,20\}$ 处报告指标,并给出 \textbf{95\% 不确定性区间}。当实验代码提供 bootstrap 置信区间时,直接采用;否则按折间方差近似为 $\pm 1.96\cdot \sigma/\sqrt{5}$(正态近似,其中 $\sigma$ 为各折标准差)。

\textbf{误差条约定}:图中的误差条均表示均值的 95\% 置信区间。表格给出的是折间标准差,便于读者自行按 $\pm 1.96\cdot \text{std}/\sqrt{5}$ 计算 CI。

% =========================================================
\section{结果与分析}

\subsection{总体性能对比}


\includefigure[width=\linewidth]{outputs/fig1_method_comparison.png}{方法对比}



{ % 使用大括号限制设置的作用范围
\setlength{\tabcolsep}{1.5pt} % 极大地减小列间距
\footnotesize % 整体缩小字体

\begin{longtable}{@{}llrrrrrrrrr@{}}
\caption{完整性能对比(按 Hit@10 排序)}\label{tab:full_results}\\
\toprule
\multirow{2}{*}{方法} & \multirow{2}{*}{类别} & \multicolumn{4}{c}{Hit Rate (\%)} & \multicolumn{4}{c}{Ranking Metrics} & \multirow{2}{*}{CI$\pm$pp} \\
\cmidrule(lr){3-6} \cmidrule(lr){7-10}
 & & @1 & @5 & @10 & @20 & P@10 & MRR & MAP & NDCG & \\
\midrule
\endfirsthead
\toprule
\multirow{2}{*}{方法} & \multirow{2}{*}{类别} & \multicolumn{4}{c}{Hit Rate (\%)} & \multicolumn{4}{c}{Ranking Metrics} & \multirow{2}{*}{CI$\pm$pp} \\
\cmidrule(lr){3-6} \cmidrule(lr){7-10}
 & & @1 & @5 & @10 & @20 & P@10 & MRR & MAP & NDCG & \\
\midrule
\endhead

M8: CLAP & 预训练 & 96.00 & \textbf{99.05} & \textbf{99.50} & 99.75 & \textbf{93.22} & 0.973 & \textbf{0.915} & \textbf{0.939} & 0.30 \\
M9: Hybrid & 预训练 & \textbf{96.10} & 99.00 & 99.45 & \textbf{99.80} & 92.75 & \textbf{0.974} & 0.910 & 0.935 & 0.25 \\
BEATs & 预训练 & 95.15 & 98.20 & 99.10 & 99.60 & 91.94 & 0.965 & 0.902 & 0.927 & 0.38 \\
Contrastive & 深度学习 & 71.75 & 83.40 & 87.80 & 92.45 & 65.96 & 0.767 & 0.608 & 0.672 & 1.42 \\
CNN & 深度学习 & 71.95 & 79.30 & 82.55 & 87.10 & 69.78 & 0.752 & 0.667 & 0.703 & 2.50 \\
\textbf{M5: DTW} & 传统 & 31.65 & 57.95 & \textbf{70.45} & 81.20 & 20.86 & 0.430 & 0.134 & 0.230 & 2.00 \\
Autoencoder & 深度学习 & 30.50 & 56.50 & 67.95 & 78.80 & 18.90 & 0.416 & 0.120 & 0.211 & 0.90 \\
M7: MultiRes & 传统 & 28.45 & 55.15 & 66.95 & 79.80 & 17.65 & 0.396 & 0.108 & 0.197 & 1.90 \\
M2: MFCC+$\Delta$ & 传统 & 27.15 & 52.75 & 65.50 & 79.20 & 17.12 & 0.382 & 0.104 & 0.190 & 2.15 \\
M6: BoAW & 传统 & 25.75 & 53.10 & 65.20 & 76.00 & 16.57 & 0.372 & 0.099 & 0.184 & 1.47 \\
M1: MFCC+Pool & 传统 & 26.40 & 52.15 & 65.00 & 78.45 & 16.75 & 0.374 & 0.100 & 0.186 & 1.80 \\
M3: LogMel & 传统 & 25.80 & 52.60 & 64.20 & 76.75 & 17.88 & 0.372 & 0.111 & 0.196 & 2.17 \\
M4: Spectral & 传统 & 17.40 & 42.90 & 55.00 & 68.65 & 11.74 & 0.281 & 0.062 & 0.129 & 2.93 \\
\bottomrule
\end{longtable}
}


\textbf{关键观察:}
\begin{enumerate}
  \item \textbf{预训练模型遥遥领先}:CLAP 达到 \textbf{99.50\% 的 Hit@10},几乎解决检索任务。512 维的音频-语言对齐嵌入来自 AudioSet 的大规模预训练,包含丰富语义信息。
  \item \textbf{29 个百分点差距}:最佳传统方法(M5: DTW,Hit@10=70.45\%)仍比 CLAP 低近 30 个百分点,凸显大规模迁移学习的价值。
  \item \textbf{DTW 优于池化方法}:在传统方法中,M5(DTW)的 Hit@10 比次优的 M7(MultiRes)\textbf{高 3.50pp}(70.45$-$66.95=3.50pp)。DTW 的序列对齐保留了全局池化丢失的时间结构。
  \item \textbf{对比学习优于基线 CNN}:监督式对比学习方法(Hit@10=87.80\%)比基线 CNN(82.55\%)高 5.25pp,说明对比学习目标能学习到更具判别性的嵌入,而不仅是交叉熵分类。
\end{enumerate}

\includefigure[width=\linewidth]{outputs/fig2_method_categories.png}{Method Categories}

\textbf{图 2}:分组层面分析。\textbf{(左)} 各类别最优方法在各指标上的分组柱状图对比。\textbf{(中)} 类别最优间的绝对差距可视化。\textbf{(右)} 各方法在各指标上的热力图,按 Hit@10 排序。

\begin{table}[H]
\centering
\caption{表 2:类别汇总}
\begin{tabular}{@{}l l c c c@{}}
\toprule
类别 & 最佳方法 & Hit@1 (\%) & Hit@10 (\%) & CI$\pm$pp \\
\midrule
传统 & M5: MFCC+DTW & 31.65 & 70.45 & 2.00 \\
深度学习 & Contrastive & 71.75 & 87.80 & 1.43 \\
预训练 & M8: CLAP & 96.00 & 99.50 & 0.30 \\
\bottomrule
\end{tabular}
\end{table}

预训练类别的 \textbf{方差最低}(CI $\approx$ 0.30pp),说明其在不同折上的表现非常稳定。深度学习方法的方差中等(CI $\approx$ 1.4--2.5pp),Contrastive 的稳定性略优于 CNN。

\subsection{超参数敏感性分析}

\includefigure[width=\linewidth]{outputs/fig3_grid_search.png}{Grid Search}

\textbf{图 3}:超参数网格搜索结果。\textbf{(左上)} 在 \code{n_mels=64, n_mfcc=20} 下,帧长(20--80ms)$\times$ 帧移(5--40ms)的 Hit@10 热力图;颜色越暖代表性能越高,最优区域在 40ms 帧长与 5--10ms 短帧移。\textbf{(右上)} \code{n_mels}(40,64,128)$\times$ \code{n_mfcc}(13,20,40)的 Hit@10 热力图。\textbf{(左下)} Step 1 网格搜索的前 10 名配置柱状图。\textbf{(右下)} 窗函数比较,Hann 略优于 Hamming 与 Boxcar。

\begin{table}[H]
\centering
\caption{表 3:最佳帧长/帧移配置}
\begin{tabular}{@{}c c c c c c@{}}
\toprule
排名 & 帧长 (ms) & 帧移 (ms) & n\_fft & hop\_length & Hit@10 (\%) \\
\midrule
1 & 40 & 5 & 882 & 110 & \textbf{68.50} \\
2 & 25 & 5 & 551 & 110 & 68.25 \\
3 & 40 & 10 & 882 & 220 & 68.25 \\
4 & 40 & 20 & 882 & 441 & 68.25 \\
5 & 80 & 20 & 1764 & 441 & 68.00 \\
\bottomrule
\end{tabular}
\end{table}

\textbf{分析:}
\begin{itemize}
  \item \textbf{帧长}:40ms 在时间分辨率(捕捉瞬态)与频率分辨率(解析谐波)之间取得最佳平衡。20ms 也具竞争力,而 80ms 的收益递减。
  \item \textbf{帧移}:更短的帧移(5--10ms)稳定优于长帧移(20--40ms),说明\textbf{高时间分辨率}对检索有益,即使计算量更大。
  \item \textbf{频率分辨率权衡}:最优的 40ms 对应约 24 Hz 的频率分辨率($\Delta f=f_s/N=22050/882\approx 25$ Hz),对于环境声音已足够,不需要更细分辨率。
\end{itemize}

\begin{table}[H]
\centering
\caption{表 4:MFCC 参数优化}
\begin{tabular}{@{}c c c c c c@{}}
\toprule
排名 & n\_mels & n\_mfcc & 帧长 (ms) & 帧移 (ms) & Hit@10 (\%) \\
\midrule
1 & 128 & 40 & 40 & 5 & \textbf{69.50} \\
2 & 64 & 40 & 40 & 5 & 69.25 \\
3 & 64 & 40 & 40 & 10 & 69.25 \\
4 & 128 & 40 & 40 & 10 & 69.00 \\
5 & 128 & 13 & 40 & 5 & 68.75 \\
\bottomrule
\end{tabular}
\end{table}

\textbf{洞见}:更高的 \code{n_mfcc}(40 vs. 13/20)带来小幅提升(+0.75pp),说明更细的倒谱细节有助于检索。然而 \code{n_mels=64} 与 \code{n_mels=128} 的差异很小,表明更细的 Mel 分辨率收益递减。

\begin{table}[H]
\centering
\caption{表 5:窗函数比较}
\begin{tabular}{@{}l c c c c@{}}
\toprule
窗函数 & Hit@1 (\%) & Hit@10 (\%) & Std (\%) & MRR@10 \\
\midrule
\textbf{Hann} & 27.40 & \textbf{67.25} & 3.14 & 0.393 \\
Hamming & 27.55 & 67.15 & 3.44 & 0.391 \\
Boxcar & 28.00 & 66.00 & 3.67 & 0.387 \\
\bottomrule
\end{tabular}
\end{table}

\textbf{Hann 与 Hamming 基本持平}(Hit@10 $\approx$ 67.25\%),Hann 的方差略低(3.14\% vs 3.44\%),两者都优于 Boxcar(66.00\%)。矩形窗(Boxcar)因截断突变导致谱泄漏,性能下降。

\subsection{消融实验}

\includefigure[width=\linewidth]{outputs/fig4_ablations.png}{Ablations}

\textbf{图 4}:对 M1(MFCC+Pool+Cos)基线的特征工程消融。\textbf{(左)} 预加重:0.97 系数 vs. 无预加重。\textbf{(中)} CMVN 变体:无、逐条语音级、全局。\textbf{(右)} Mel 标度公式:Slaney vs. HTK。误差条表示 95\% CI。

\begin{table}[H]
\centering
\caption{表 6:预加重消融}
\begin{tabular}{@{}l c c c c@{}}
\toprule
配置 & Hit@1 (\%) & Hit@10 (\%) & $\Delta$ Hit@1 & $\Delta$ Hit@10 \\
\midrule
无预加重 & 26.40 & \textbf{65.00} & --- & --- \\
预加重($\alpha=0.97$) & 27.65 & 63.65 & +1.25 & \textbf{-1.35} \\
\bottomrule
\end{tabular}
\end{table}

\textbf{结论}:预加重 \textbf{提高 Hit@1}(+1.25pp)但 \textbf{降低 Hit@10}(-1.35pp)。这说明强调高频有助于第一名精度,但会引入噪声,影响更深层的召回。对环境声音(不同于语音)而言,预加重可能过度放大背景噪声。

\begin{table}[H]
\centering
\caption{表 7:CMVN 消融}
\begin{tabular}{@{}l c c c c@{}}
\toprule
配置 & Hit@1 (\%) & Hit@10 (\%) & $\Delta$ Hit@1 & $\Delta$ Hit@10 \\
\midrule
无 CMVN & 26.40 & 65.00 & --- & --- \\
逐条语音 CMVN & 2.05 & 13.20 & -24.35 & \textbf{-51.80} \\
\textbf{全局 CMVN} & \textbf{31.75} & \textbf{69.55} & \textbf{+5.35} & \textbf{+4.55} \\
\bottomrule
\end{tabular}
\end{table}

\textbf{关键发现}:\textbf{全局 CMVN 显著提升}(Hit@10 +4.55pp),因为它移除了数据集级的信道偏差,同时保留了相对特征关系。相反,\textbf{逐条语音级 CMVN 灾难性失败}(-51.80pp)。

\textbf{原因分析}:失败源于逐条 CMVN 与均值/标准差池化的\textbf{方法不兼容},而非逐条归一化本身的问题:
\begin{enumerate}
  \item 逐条 CMVN 将每个特征维度在单条音频内归一化到均值$\approx 0$、标准差$\approx 1$
  \item 均值/标准差池化随后把这些统计量本身作为嵌入
  \item 结果:所有嵌入近似为 $[0,0,\dots,1,1,\dots]$(退化)
  \item 嵌入几乎相同,检索退化到近似随机
\end{enumerate}
这提醒我们:两个单独看起来合理的操作组合在一起也可能产生病态结果。

\begin{table}[H]
\centering
\caption{表 8:Mel 标度公式消融}
\begin{tabular}{@{}l c c c c@{}}
\toprule
配置 & Hit@1 (\%) & Hit@10 (\%) & $\Delta$ Hit@1 & $\Delta$ Hit@10 \\
\midrule
Slaney & 26.40 & 65.00 & --- & --- \\
HTK & 26.20 & 65.35 & -0.20 & +0.35 \\
\bottomrule
\end{tabular}
\end{table}

Slaney 与 HTK 的选择影响 \textbf{可以忽略}($\pm$0.35pp),与其在多数频率范围内数学上的相似性一致。

\subsection{鲁棒性分析}

\includefigure[width=\linewidth]{outputs/fig5_robustness.png}{Robustness}

\textbf{图 5}:在 M1 基线下的多种音频扰动鲁棒性。\textbf{(左上)} 20dB、10dB、0dB SNR 的高斯噪声。\textbf{(右上)} 音量缩放 $\pm$6dB。\textbf{(左下)} 语速扰动 0.9$\times$ 与 1.1$\times$。\textbf{(右下)} 变调 $\pm$1 半音以及 10--20\% 的随机时间偏移。

\begin{table}[H]
\centering
\caption{表 9:鲁棒性结果}
\begin{tabular}{@{}l c c c@{}}
\toprule
条件 & Hit@1 (\%) & Hit@10 (\%) & 相对 Clean 的 $\Delta$ Hit@1 \\
\midrule
\textbf{Clean} & 26.40 & 65.00 & --- \\
噪声 20dB & 22.40 & 59.55 & -15.2\% \\
噪声 10dB & 17.05 & 52.65 & -35.4\% \\
噪声 0dB & 8.15 & 33.70 & \textbf{-69.1\%} \\
音量 +6dB & 23.40 & 61.80 & -11.4\% \\
音量 -6dB & 22.85 & 61.80 & -13.4\% \\
语速 0.9$\times$ & 21.90 & 65.50 & -17.0\% \\
语速 1.1$\times$ & 14.50 & 46.95 & \textbf{-45.1\%} \\
变调 -1 & 23.10 & 64.55 & -12.5\% \\
变调 +1 & 24.85 & 63.60 & -5.9\% \\
时间偏移 0.1 & 26.25 & 65.00 & -0.6\% \\
时间偏移 0.2 & 26.15 & 64.95 & -0.9\% \\
\bottomrule
\end{tabular}
\end{table}

\textbf{关键发现:}
\begin{enumerate}
  \item \textbf{噪声是主导性退化因素}:在 0dB SNR(信号与噪声功率相等)时,Hit@1 相对干净条件下降 69.1\%。MFCC 捕捉的是谱包络,对覆盖全频段的加性噪声较敏感。
  \item \textbf{语速扰动(1.1$\times$)比变调更致命}:加速 10\% 造成 45.1\% 的 Hit@1 相对下降,而 $\pm$1 半音变调仅下降 5.9--12.5\%。这是因为语速变化同时压缩时间与频谱结构,而变调主要影响谐波,MFCC 在一定程度上可去相关。
  \item \textbf{时间偏移几乎无影响}:10--20\% 的随机偏移导致退化 $<1\%$,说明帧级特征与池化对对齐误差鲁棒。
  \item \textbf{音量变化对称退化}:+6dB 与 -6dB 均带来约 12\% 的下降,表明余弦距离(忽略幅度)无法完全抵消对数 Mel 特征的非线性幅度效应。
\end{enumerate}

\subsection{效率分析}

\includefigure[width=\linewidth]{outputs/fig6_efficiency.png}{Efficiency}

\textbf{图 6}:传统方法 M1-M7 的效率-准确率权衡。\textbf{(左)} Hit@10 与检索延迟(ms/查询)的帕累托前沿;前沿上的方法代表最佳权衡。\textbf{(中)} 特征提取与检索时间分解。\textbf{(右)} 候选库嵌入的内存占用。

\begin{table}[H]
\centering
\caption{表 10:效率指标}
\begin{tabular}{@{}l c c c c c c@{}}
\toprule
方法 & 特征 (ms) & 检索 (ms) & QPS & 内存 (MB) & 维度 & Hit@10 \\
\midrule
M1: MFCC+Pool & 6.61$\pm$1.29 & 6.96$\pm$0.96 & 143.7 & 0.24 & 40 & 65.00 \\
M2: MFCC+$\Delta$ & 8.71$\pm$1.04 & 9.10$\pm$1.11 & 109.9 & 0.73 & 120 & 65.50 \\
M3: LogMel & 6.75$\pm$0.95 & 7.18$\pm$1.35 & 139.3 & 1.56 & 256 & 64.20 \\
M4: Spectral & 16.35$\pm$2.32 & 17.19$\pm$2.85 & 58.2 & 0.17 & 28 & 55.00 \\
\textbf{M5: DTW} & 4.91$\pm$0.58 & \textbf{67.35$\pm$6.64} & \textbf{14.8} & 17.14 & 13 & \textbf{70.45} \\
\textbf{M6: BoAW} & 5.79$\pm$1.57 & 5.95$\pm$0.68 & \textbf{168.0} & 0.78 & 128 & 65.20 \\
M7: MultiRes & 13.18$\pm$2.41 & 13.58$\pm$1.80 & 73.6 & 0.49 & 80 & 66.95 \\
\bottomrule
\end{tabular}
\end{table}

\textbf{帕累托最优方法:}
\begin{enumerate}
  \item \textbf{M6(BoAW)}:最高吞吐 \textbf{168 QPS},同时保持竞争性精度(Hit@10=65.20\%)。直方图表示使卡方距离计算高效。
  \item \textbf{M1(MFCC+Pool)}:在 \textbf{143.7 QPS} 下取得良好平衡,并有极小内存占用(0.24 MB),适合资源受限部署。注意:M4 内存更小(0.17 MB),但精度更低(55.00\%)。
  \item \textbf{M5(DTW)}:精度最高(70.45\%),但 \textbf{吞吐最低(14.8 QPS)},因为其序列对齐复杂度为 $O(nm)$。DTW 的 67ms 检索延迟适合离线或小规模场景。
\end{enumerate}

\textit{注}:QPS 按检索时间计算(不含特征提取)。若计入特征提取,端到端吞吐更低。例如 M1 的真实端到端吞吐约为 73 QPS($1000/(6.61+6.96)$ ms)。

\textbf{内存效率}:M5 的候选库存储可变长度 MFCC 序列,需 \textbf{17.14 MB},比 M1 的定长嵌入(0.24 MB)大 71 倍。大规模部署中,池化方法能显著节省存储。

\subsection{融合与两阶段检索}

\includefigure[width=\linewidth]{outputs/fig7_fusion_twostage.png}{Fusion and Two-Stage}

\textbf{图 7}:高级检索策略。\textbf{(左)} M1、M3、M4 的晚融合(学习权重)。\textbf{(中)} 互惠排序融合(RRF)。\textbf{(右)} 两阶段检索:M1 快速召回后用 DTW 重排,展示 Hit@10 与延迟随候选池大小 $N$ 的变化。

\begin{table}[H]
\centering
\caption{表 11:融合结果}
\begin{tabular}{@{}l c c c@{}}
\toprule
方法 & Hit@1 (\%) & Hit@10 (\%) & MRR@10 \\
\midrule
最佳单方法(M2) & 27.15 & 65.50 & 0.382 \\
晚融合(M1:0.5, M3:0.5) & 27.80 & 66.40 & 0.389 \\
排名融合(RRF) & 27.85 & 66.25 & 0.388 \\
\bottomrule
\end{tabular}
\end{table}

晚融合与排名融合仅带来小幅提升(Hit@10 +0.75--0.90pp),说明 M1-M4 捕获的信息部分冗余。

\begin{table}[H]
\centering
\caption{表 12:两阶段检索(M1 $\rightarrow$ DTW 重排)}
\begin{tabular}{@{}c c c c c@{}}
\toprule
$N$ & Hit@10 (\%) & 延迟 (ms) & 加速比 & 精度保留率 \\
\midrule
20 & 69.20 & 21.71 & \textbf{2.82$\times$} & 98.2\% \\
50 & \textbf{71.70} & 22.86 & 2.68$\times$ & \textbf{101.8\%} \\
100 & 71.00 & 24.77 & 2.47$\times$ & 100.8\% \\
200 & 71.35 & 28.59 & 2.14$\times$ & 101.3\% \\
500 & 70.45 & 40.06 & 1.53$\times$ & 100.0\% \\
1000 & 70.50 & 59.19 & 1.03$\times$ & 100.1\% \\
1600 & 70.45 & 82.13 & 0.74$\times$ & 100.0\% \\
\bottomrule
\end{tabular}
\end{table}

\includefigure[width=\linewidth]{outputs/fig9_twostage_pareto.png}{Two-Stage Pareto}

\textbf{图 9}:两阶段检索的帕累托前沿,展示 $N$ 从 20 到 1600 时的准确率-延迟权衡。$N=50$ 处取得最优工作点,\textbf{精度保留 101.8\%}(略高于全量 DTW),同时 \textbf{2.68$\times$ 加速}。

\textbf{关键洞见}:两阶段在 $N=50$ 时 \textbf{优于全量 DTW}(71.70\% vs 70.45\%)且快 2.68$\times$。这一看似反直觉的结果可能源于 M1 的粗召回阶段筛掉了 DTW 容易误排的困难负例。粗到细策略因此同时带来效率与准确率收益。

\textit{注意事项}:(1)相比全量 DTW 的 1.25pp 提升幅度不大,可能落在测量噪声内,若要更强结论需重复实验并进行显著性检验。(2)效率实验(表 10:DTW 检索 67.35ms)与两阶段实验(表 12:$N=1600$ 基线 82.13ms)的计时存在差异,可能由不同脚本或候选规模造成。加速比应理解为近似值。

\subsection{部分查询分析}

\includefigure[width=\linewidth]{outputs/fig8_partial_query.png}{Partial Query}

\textbf{图 8}:查询时长对检索性能的影响。横轴为 0.5s 到 5s 的查询片段长度,短查询从音频中心截取。随着查询长度减少,性能平滑下降。

\begin{table}[H]
\centering
\caption{表 13:部分查询结果}
\begin{tabular}{@{}c c c c c c@{}}
\toprule
时长 (s) & Hit@1 (\%) & Hit@5 (\%) & Hit@10 (\%) & MRR@10 & 保留率 \\
\midrule
0.5 & 21.30 & 43.35 & 55.75 & 0.309 & 85.8\% \\
1.0 & 22.70 & 45.20 & 58.20 & 0.325 & 89.5\% \\
2.0 & 24.45 & 49.55 & 62.25 & 0.352 & 95.8\% \\
3.0 & 25.30 & 51.80 & 63.85 & 0.364 & 98.2\% \\
\textbf{5.0} & \textbf{26.40} & \textbf{52.15} & \textbf{65.00} & \textbf{0.374} & 100.0\% \\
\bottomrule
\end{tabular}
\end{table}

\textbf{分析}:随着查询长度缩短,性能以较缓的方式下降:
\begin{itemize}
  \item 0.5s 查询仍保留 85.8\% 的全长 Hit@10
  \item 2.0s 查询保留 95.8\%,对多数实际应用已足够
\end{itemize}
这说明环境声音在短片段中已包含足够的判别信息,支持实时的“部分音频检索”。

\subsection{跨折稳定性}

\includefigure[width=\linewidth]{outputs/fig10_fold_variance.png}{Fold Variance}

\textbf{图 10}:交叉验证稳定性分析。\textbf{(左)} 13 种方法在 5 折上的 Hit@10 均值 $\pm$ 标准差,按均值排序。\textbf{(中)} 变异系数(CV = std/mean $\times$ 100\%)反映相对稳定性。\textbf{(右)} 折内分布箱线图。

\begin{quote}
\textbf{关于深度模型评测的说明}:所有深度学习模型(CNN、Autoencoder、Contrastive)均采用严格的 5 折交叉验证,并为每个折训练独立的模型检查点。每折结果都是真正的样本外表现。
\end{quote}

\begin{table}[H]
\centering
\caption{表 14:折间方差分析}
\begin{tabular}{@{}l l c c c c c@{}}
\toprule
方法 & Hit@10 均值$\pm$Std & CV (\%) & 最小折 & 最大折 \\
\midrule
M8: CLAP & 99.50 $\pm$ 0.35 & \textbf{0.4} & 99.00 & 100.00 \\
M9: Hybrid & 99.45 $\pm$ 0.29 & \textbf{0.3} & 99.00 & 99.75 \\
BEATs & 99.10 $\pm$ 0.41 & 0.4 & 98.50 & 99.50 \\
Contrastive & 87.80 $\pm$ 1.68 & 1.9 & 86.50 & 91.00 \\
CNN & 82.55 $\pm$ 2.99 & 3.6 & 77.25 & 86.25 \\
M5: DTW & 70.45 $\pm$ 2.25 & 3.2 & 67.75 & 74.25 \\
M7: MultiRes & 66.95 $\pm$ 2.16 & 3.2 & 64.25 & 69.25 \\
M4: Spectral & 55.00 $\pm$ 3.48 & \textbf{6.3} & 50.75 & 59.25 \\
\bottomrule
\end{tabular}
\end{table}

\textbf{稳定性结论:}
\begin{enumerate}
  \item \textbf{预训练模型最稳定}:CLAP 与 Hybrid 的 CV $<0.5\%$,说明其表征在不同数据划分上具有一致的泛化能力。
  \item \textbf{深度学习方法方差中等}:Contrastive(87.80\%)与 CNN(82.55\%)的方差相当(CV $\approx$ 2--4\%),这是在小数据集上训练神经网络的常见现象。
  \item \textbf{传统方法的 CV 约 3--4\%}:M1-M7 方差中等且稳定,说明手工特征在不同分布下表现可预测。
  \item \textbf{M4(Spectral)最不稳定}:其 6.3\% CV 反映了低维频谱统计特征的判别能力有限。
\end{enumerate}

% =========================================================
\section{讨论}

\subsection{为什么 DTW 优于池化方法}

在传统方法中,M5(MFCC+DTW)比池化方法 \textbf{高 3--5pp Hit@10}。其优势来自 DTW 能够:
\begin{enumerate}
  \item \textbf{保留时间结构}:环境声音往往具有特征性时间模式(起音-持续-衰减包络、重复节奏)。全局池化会折叠时间轴,丢失这些信息。
  \item \textbf{处理不同速率事件}:DTW 的弹性对齐可适配查询与候选中速度不同或起始不同的声音。
  \item \textbf{利用完整序列信息}:池化丢弃细粒度的帧级变化,而 DTW 关注完整特征轨迹。
\end{enumerate}
4.5$\times$ 的延迟成本(67ms vs 池化方法约 15ms)来自 DTW 的 $O(nm)$ 序列对齐复杂度。

\subsection{为什么 CLAP 占据优势}

CLAP 的 \textbf{99.50\% Hit@10} 体现了预训练音频-语言模型的优势:
\begin{enumerate}
  \item \textbf{预训练规模}:CLAP 在 AudioSet(200 万样本)及额外音频-文本对上训练,训练数据规模比 ESC-50 大 1000$\times$。
  \item \textbf{语义监督}:音频-文本对比学习促使嵌入捕捉语义类别(如“狗叫” vs “汽车发动机”),与检索目标直接对齐。
  \item \textbf{模型容量}:HTSAT 主干(分层音频 Transformer)的表征能力强于简单 CNN 或手工特征。
  \item \textbf{迁移学习}:预训练表征很好地迁移到 ESC-50 的环境声音类别上,这些类别与 AudioSet 的本体有高度重叠。
\end{enumerate}

\subsection{误差分析}

对失败样例的观察揭示了系统性模式:

\textbf{易混淆类别(传统方法)}:
\begin{itemize}
  \item \textbf{雨声 vs. 水滴声}:均为宽带脉冲声,频谱相似
  \item \textbf{直升机 vs. 电锯}:均呈现旋转机械的周期性谐波模式
  \item \textbf{键盘打字 vs. 时钟滴答}:均为具有相似攻击特性的脉冲声
\end{itemize}

\textbf{鲁棒类别}:
\begin{itemize}
  \item \textbf{狗叫}:具有显著谐波结构与时间包络
  \item \textbf{警报}:频率扫动特征明显,易于 MFCC 捕捉
  \item \textbf{雷声}:低频轰鸣与长衰减特征独特
\end{itemize}

预训练模型大多能通过语义理解消除这些混淆,而非仅依赖声学相似性。

% =========================================================
\section{结论}

\subsection{关键发现}

基于实验结果,我们总结出以下结论:
\begin{enumerate}
  \item \textbf{最佳传统方法}:M5(MFCC+DTW)达到 \textbf{70.45\% Hit@10},因时间对齐比池化方法高 3--6pp。
  \item \textbf{最佳总体方法}:M8(CLAP)达到 \textbf{99.50\% Hit@10},相对传统方法提升 29pp。
  \item \textbf{关键特征工程}:\textbf{全局 CMVN 使 Hit@10 提升 +4.55pp},而逐条 CMVN 灾难性失败(-51.80pp)。
  \item \textbf{最优超参数}:40ms 帧长、5--10ms 帧移、Hann 窗、n\_mfcc=40、n\_mels=128。
  \item \textbf{效率-准确率权衡}:两阶段检索(M1$\rightarrow$DTW,$N=50$)在 \textbf{2.68$\times$ 加速} 下保持 \textbf{101.8\% 精度保留率}。
\end{enumerate}

\subsection{未来工作}
未来我们从这几个方面继续探索与尝试:
\begin{enumerate}
  \item \textbf{领域自适应}:在特定领域音频上微调 CLAP(如工业声音、野外动物)可能进一步提升性能。
  \item \textbf{混合架构}:将 CLAP 嵌入与手工特征进行可学习融合,捕捉互补信息。
  \item \textbf{近似最近邻}:用 HNSW 或 Faiss 替代穷举搜索,实现百万级检索。
  \item \textbf{跨模态检索}:利用 CLAP 的联合嵌入空间扩展到文本到音频检索。
\end{enumerate}

% =========================================================
\newpage
\part{任务二:声音分类}

本部分的基础信号处理算法(FFT、STFT、MFCC 等)与任务一相同,此处不再赘述。以下直接进入分类实验。

\section{帧长/帧移超参数实验}

为研究不同帧长(n\_fft)和帧移(hop\_length)对分类性能的影响,我们使用 ResNet18 模型在 Mel 频谱图和 MFCC 特征上进行了系统性实验。

\subsection{实验设置}

\begin{itemize}
    \item \textbf{模型}:ResNet18(ImageNet 预训练)
    \item \textbf{训练轮数}:30 epochs
    \item \textbf{学习率}:$1 \times 10^{-3}$
    \item \textbf{批大小}:32
    \item \textbf{数据增强}:SpecAugment
    \item \textbf{评估指标}:Top-1 准确率
\end{itemize}

\subsection{实验结果}

\begin{table}[H]
    \centering
    \caption{不同帧长/帧移配置下 ResNet18 的分类精度}
    \label{tab:frame_length_classification}
    \vspace{0.2cm}
    \begin{tabular}{cccc}
        \toprule
        \textbf{帧长 (n\_fft)} & \textbf{帧移 (hop\_length)} & \textbf{Mel 准确率 (\%)} & \textbf{MFCC 准确率 (\%)} \\
        \midrule
        1024 & 256 & 76.25 & 68.75 \\
        \textbf{2048} & \textbf{512} & \textbf{81.75} & \textbf{70.75} \\
        4096 & 1024 & 78.75 & 64.75 \\
        8192 & 2048 & 71.00 & 52.00 \\
        \bottomrule
    \end{tabular}
\end{table}

\subsection{分析与讨论}

\subsubsection{最优配置}
实验结果表明,\textbf{n\_fft=2048, hop\_length=512} 是最优配置,Mel 特征达到 \textbf{81.75\%} 的最高准确率。这一配置在时间分辨率和频率分辨率之间取得了良好平衡。

\subsubsection{特征类型对比}
在所有帧长配置下,Mel 频谱图特征均优于 MFCC 特征,差距约为 7--19 个百分点。这是因为:
\begin{itemize}
    \item Mel 频谱图保留了更完整的频谱信息
    \item MFCC 的 DCT 变换丢弃了部分对分类有用的细节
    \item ResNet 的卷积结构更适合处理 2D 频谱图输入
\end{itemize}

\subsubsection{帧长影响规律}
\begin{itemize}
    \item \textbf{帧长过小(1024)}:频率分辨率不足,难以区分相近频率成分
    \item \textbf{帧长适中(2048)}:时频分辨率平衡,性能最优
    \item \textbf{帧长过大(4096--8192)}:时间分辨率下降,丢失瞬态信息,准确率显著下降
\end{itemize}

值得注意的是,当 n\_fft=8192 时,MFCC 准确率骤降至 52\%,仅略高于随机猜测(2\%),说明过长的帧长对 MFCC 特征的影响尤为严重。

\section{深度学习模型实验}
\subsection{实验设置}

我们评估了两种在 AudioSet 上预训练的先进架构:
\begin{itemize}
    \item \textbf{BEATs (Iter3+):} 一种基于 Transformer 且使用声学标记器的模型。
    \item \textbf{PANNs (CNN14):} 一种针对音频优化的类 ResNet 标准 CNN 架构。
\end{itemize}

\subsubsection{基础超参数}
为确保公平比较,两种模型共享以下配置:
\begin{itemize}
    \item \textbf{学习率 (LR):} $1 \times 10^{-4}$ (AdamW)
    \item \textbf{权重衰减 (Weight Decay):} $1 \times 10^{-4}$
    \item \textbf{批大小 (Batch Size):} 64
    \item \textbf{训练轮数 (Epochs):} 50
\end{itemize}

\subsection{消融实验与对比结果}

我们通过消融实验来隔离增强策略和解冻策略的影响。表 \ref{tab:detailed_results} 汇总了两种架构的关键发现。

\begin{table}[H]
    \centering
    \caption{BEATs 与 CNN14 架构下不同增强策略的详细性能对比。}
    \label{tab:detailed_results}
    \vspace{0.2cm}
    \begin{tabular}{llcc}
        \toprule
        \textbf{模型} & \textbf{增强策略} & \textbf{骨干状态} & \textbf{准确率 (\%)} \\
        \midrule
        \midrule
        \multirow{5}{*}{\textbf{BEATs}} 
         & 无 (基准线) & 冻结 & 94.50\% \\
         & 仅波形增强 (Waveform) & 冻结 & 93.50\% \\
         & 仅 SpecAugment & 第 10 轮解冻 & 94.00\% \\
         & 仅 Mixup & 第 10 轮解冻 & 95.50\% \\
         & \textbf{SpecAugment + Mixup} & \textbf{第 10 轮解冻} & \textbf{96.50\%} \\
        \midrule
        \midrule
        \multirow{1}{*}{\textbf{CNN14}} 
         
         & \textbf{SpecAugment + Mixup} & \textbf{第 10 轮解冻} & \textbf{92.75\%} \\
        \bottomrule
    \end{tabular}
\end{table}

\subsection{分析与讨论}

\subsubsection{“增强 + 解冻”策略的普适性}
本研究的一个核心发现是:数据增强与模型容量之间的交互作用是\textbf{与架构无关的}。

\begin{itemize}
    \item \textbf{“冻结”陷阱:} 对于 CNN14 和 BEATs 而言,在骨干网络冻结时应用 SpecAugment(遮蔽)或波形增强(失真)都会导致性能下降。
    \begin{itemize}
        \item 对于 \textbf{CNN14},冻结的卷积滤波器旨在寻找特定的频谱纹理。遮蔽破坏了这些纹理,导致固定滤波器输出零或噪声。
        \item 对于 \textbf{BEATs},冻结的标记器会误读失真的音频,从而生成错误的语义标记(Tokens)。
    \end{itemize}
    \item \textbf{解决方案:} 在经过预热期(Warmup)后解冻骨干网络,使得两种架构都能进行自适应。CNN14 更新了其滤波器以增强对部分遮挡的鲁棒性,而 BEATs 则更新了其注意力机制以处理标记噪声。
\end{itemize}


\subsubsection{为什么 BEATs 优于 CNN14 尽管使用了相同的优化策略,BEATs 的准确率仍显著更高。}
\begin{itemize}
    \item \textbf{语义 vs. 结构:} CNN14 依赖于局部的时频谱特征(边缘、纹理)。BEATs 使用掩码建模目标(类 BERT),能够捕获更高层级的语义信息。
    \item \textbf{鲁棒性:} 相比于通常需要大量数据来从头学习不变特征的 CNN,BEATs 的 Transformer 架构结合 Mixup 证明了其在 ESC-50 这种小规模数据集(2000 个样本)上具有更强的泛化能力。
\end{itemize}

此外,CNN14用的是手写的dspcore魔改后的版本,如果直接原本端到端的版本,用库函数提取特征,可能可以达到更好的效果。

\subsubsection{超参数敏感性}
CNN14 的调优过程映证了 BEATs 的实验结果,确认了以下参数的敏感性:
\begin{itemize}
    \item \textbf{SpecAugment 宽度:} CNN14 对频率遮蔽比 BEATs 更敏感。我们必须减小 CNN14 的 \texttt{freq\_drop\_width},这可能是因为 CNN 高度依赖连续的频率谐波。
    \item \textbf{解冻时机:} 两种模型都受益于“预热”期(前 10 轮),即先仅训练分类头。立即解冻会导致两种模型的训练过程均出现不稳定。
\end{itemize}

\subsection{结论}
我们验证了一套适用于 CNN 和 Transformer 架构的鲁棒训练流水线。虽然盲目的增强会损害冻结的预训练模型,但 \textbf{SpecAugment、Mixup 与延迟解冻} 的组合能够释放显著的性能提升。

该策略将 CNN14 的准确率提升至 \textbf{92.75\%},并将 BEATs 提升至 \textbf{96.50\%} 的领先水平。结果表明,在 ESC-50 等数据受限的场景下,只要配合正确的正则化策略,基于声学标记器的 Transformer 模型(BEATs)比传统 CNN 具有更优越的表示能力。

\section{大模型对比与极致优化}
\subsection{Linear Probing 方法}

\subsubsection{核心思想}
冻结预训练 CLAP 模型作为特征提取器,仅训练一个轻量级线性分类器进行音频分类。

\subsubsection{核心策略}

\begin{tcolorbox}[colback=gray!10, colframe=black!50, title=三大核心策略]
\begin{enumerate}
    \item 冻结 CLAP 编码器(保留预训练知识)
    \item 使用 5-fold 交叉验证(充分利用数据)
    \item 训练简单线性分类器(快速高效)
\end{enumerate}
\end{tcolorbox}

\subsubsubsection{Linear Probing 方法特点}

\begin{table}[h]
\centering
\begin{tabular}{lll}
\toprule
\textbf{特点} & \textbf{说明} & \textbf{优势} \\
\midrule
轻量级 & 只训练一个线性层 & 训练速度快 \\
高效 & 特征提取一次完成 & 节省计算资源 \\
稳定 & 保留预训练知识 & 不易过拟合 \\
基线 & 作为其他方法的基准 & 便于对比 \\
\bottomrule
\end{tabular}
\caption{Linear Probing 方法特点}
\end{table}

\subsubsection{实现流程}

\begin{enumerate}
    \item \textbf{数据划分}:采用 5-Fold 交叉验证,Fold 1-4 为训练集(1600样本),Fold 5 为测试集(400样本)
    \item \textbf{模型冻结}:加载 CLAP 模型后,冻结所有参数(\code{requires\_grad=False}),设置为评估模式
    \item \textbf{特征提取}:使用冻结的 CLAP 编码器提取 512 维音频嵌入,特征只需提取一次
    \item \textbf{分类器训练}:训练单层线性分类器(参数量仅 25,650),使用 Adam 优化器,学习率 0.001
\end{enumerate}

\begin{table}[h]
\centering
\caption{Linear Probing 训练配置}
\begin{tabular}{lll}
\toprule
\textbf{配置项} & \textbf{值} & \textbf{说明} \\
\midrule
特征维度 & 512 & CLAP 音频嵌入维度 \\
分类器 & 单层线性 & Input(512) $\rightarrow$ Output(50) \\
优化器 & Adam & 学习率 0.001 \\
训练轮数 & 50 & 通常 10-20 轮即可收敛 \\
\bottomrule
\end{tabular}
\end{table}

\subsubsection{预期性能}

\begin{tcolorbox}[colback=green!10, colframe=green!50!black, title=性能对比]
\begin{itemize}
    \item Zero-shot (基础): 93.90\%
    \item Zero-shot (提示词优化): 95.00\%
    \item Linear Probing: 96.50-97.50\% 
\end{itemize}
\end{tcolorbox}

\newpage

\subsection{Ultimate Optimization 方法}

\subsubsection{核心思想}
通过多维度集成策略,最大化利用预训练 CLAP 模型的特征提取能力。

\subsubsection{核心策略}

\begin{tcolorbox}[colback=gray!10, colframe=black!50, title=五大核心策略]
\begin{enumerate}
    \item 极强的 TTA(20次增强)
    \item 训练集和测试集都使用强增强
    \item Label Smoothing
    \item 多尺度特征融合
    \item 自集成(训练多个模型并投票)
\end{enumerate}
\end{tcolorbox}

\subsubsection{技术实现详解}

\begin{table}[H]
\centering
\caption{Ultimate Optimization 各技术详解}
\begin{tabular}{@{}llp{6cm}@{}}
\toprule
\textbf{技术} & \textbf{配置} & \textbf{原理} \\
\midrule
\textbf{超强 TTA} & 训练10次/测试20次 & 对同一音频多次随机增强采样,计算 mean/std/max/min 四种统计量拼接为 4$\times$512=2048 维特征 \\
\textbf{Label Smoothing} & $\epsilon=0.1$ & 软化标签,目标类概率为 0.9,其他类均分 0.1,防止过拟合 \\
\textbf{增强分类器} & 3层全连接 & 2048$\rightarrow$512$\rightarrow$256$\rightarrow$50,含 BatchNorm 和 Dropout(0.4/0.3) \\
\textbf{训练优化} & AdamW + Cosine & 权重衰减 1e-4,余弦退火学习率,Early Stopping (patience=20) \\
\textbf{模型集成} & 3个模型 & 不同随机种子训练,预测概率加权平均 \\
\bottomrule
\end{tabular}
\end{table}

\begin{table}[H]
\centering
\caption{集成方法对比}
\begin{tabular}{llll}
\toprule
\textbf{方法} & \textbf{计算方式} & \textbf{优点} & \textbf{适用场景} \\
\midrule
简单平均 & mean(probs) & 简单有效 & 模型性能相近时 \\
加权平均 & $\Sigma(w_i \cdot probs_i)$ & 重视好模型 & 模型性能有差异时 \\
多数投票 & mode(preds) & 鲁棒性强 & 需要硬决策时 \\
\bottomrule
\end{tabular}
\end{table}

\subsubsection{性能提升路径}

\begin{tcolorbox}[colback=blue!10, colframe=blue!50!black, title=渐进式优化]
\begin{enumerate}
    \item \textbf{阶段1: 基础优化} - 添加简单 TTA (5x)
    \item \textbf{阶段2: 特征增强} - 多尺度特征融合
    \item \textbf{阶段3: 模型增强} - 更深的分类器网络
    \item \textbf{阶段4: 集成优化} - 模型集成 + Label Smoothing
\end{enumerate}
\textbf{最终性能}: 97\% $\rightarrow$ 98\%
\end{tcolorbox}

\newpage

\subsection{两种方法对比}

\subsubsection{方法对比表}

\begin{table}[h]
\centering
\begin{tabular}{lll}
\toprule
\textbf{维度} & \textbf{Linear Probing} & \textbf{Ultimate Optimization} \\
\midrule
准确率 & 95-96\% & 98\%+ \\
训练时间 & $\sim$30s& $\sim$5min \\
特征增强 & 无 & 20x TTA + 多尺度 \\
分类器 & 单层线性 & 3层全连接 \\
正则化 & 无 & Label Smoothing + Dropout \\
模型集成 & 单模型 & 3模型集成 \\
参数量 & 25K & 395K \\
适用场景 & 快速基线 & 极致性能 \\
\bottomrule
\end{tabular}
\caption{两种方法全面对比}
\end{table}

\subsubsection{结构对比}

\begin{table}[h]
\centering
\begin{tabular}{lll}
\toprule
\textbf{模型} & \textbf{结构} & \textbf{参数量} \\
\midrule
Linear Probe & Input $\rightarrow$ Linear $\rightarrow$ Output & 25,650 \\
Enhanced (Ultimate) & Input $\rightarrow$ 512 $\rightarrow$ 256 $\rightarrow$ Output & 395,450 \\
\bottomrule
\end{tabular}
\caption{分类器结构对比}
\end{table}

\subsubsection{性能提升分析}

从 Linear Probing 到 Ultimate Optimization 的性能提升主要来自:

\begin{enumerate}
    \item \textbf{TTA增强} (+1.5\%): 20次增强采样显著提高特征鲁棒性
    \item \textbf{多尺度特征融合} (+0.5\%): mean/std/max/min 统计特征提供更丰富的表示
    \item \textbf{深层分类器} (+0.3\%): 3层网络更好地拟合复杂决策边界
    \item \textbf{Label Smoothing} (+0.4\%): 防止过拟合,提高泛化能力
    \item \textbf{模型集成} (+0.8\%): 3个模型的预测融合降低方差
\end{enumerate}

% =========================================================
\section{将分类模型用于检索任务}

根据作业要求,我们将训练好的分类模型应用于声音检索任务,以对比有无机器学习的效果。

\subsection{方法说明}

将分类模型用于检索的核心思路是:\textbf{去掉最后的分类层,将倒数第二层的特征向量作为音频的嵌入表示},然后计算查询与候选库之间的余弦相似度进行检索。

具体实现方式:
\begin{itemize}
    \item \textbf{CNN}:使用在 ESC-50 上有监督训练的 5 层卷积网络,提取全局平均池化后的特征向量
    \item \textbf{Contrastive}:使用监督对比学习(SupCon)目标训练的 CNN,直接优化嵌入空间的类内紧凑性和类间分离性
    \item \textbf{BEATs/CLAP}:使用预训练模型的音频编码器提取嵌入向量
\end{itemize}

\subsection{有无机器学习的效果对比}

\begin{table}[H]
\centering
\caption{分类模型用于检索:有无机器学习的效果对比}
\label{tab:ml_vs_no_ml}
\begin{tabular}{@{}llcccc@{}}
\toprule
\textbf{方法} & \textbf{类型} & \textbf{Hit@1 (\%)} & \textbf{Hit@10 (\%)} & \textbf{MRR} & \textbf{相比DTW提升} \\
\midrule
\multicolumn{6}{c}{\textit{传统方法(无机器学习)}} \\
\midrule
M1: MFCC+Pool & 手工特征 & 26.40 & 65.00 & 0.374 & -5.45pp \\
M5: MFCC+DTW & 手工特征+对齐 & 31.65 & 70.45 & 0.430 & 基线 \\
M6: BoAW & 无监督聚类 & 25.75 & 65.20 & 0.372 & -5.25pp \\
\midrule
\multicolumn{6}{c}{\textit{有监督机器学习}} \\
\midrule
CNN & 有监督分类 & 71.95 & 82.55 & 0.752 & \textbf{+12.10pp} \\
Contrastive & 对比学习 & 71.75 & 87.80 & 0.767 & \textbf{+17.35pp} \\
\midrule
\multicolumn{6}{c}{\textit{预训练大模型}} \\
\midrule
BEATs & 预训练Transformer & 95.15 & 99.10 & 0.965 & \textbf{+28.65pp} \\
CLAP & 预训练多模态 & 96.00 & 99.50 & 0.973 & \textbf{+29.05pp} \\
\bottomrule
\end{tabular}
\end{table}

\subsection{分析与讨论}

\subsubsection{机器学习带来的显著提升}

从表 \ref{tab:ml_vs_no_ml} 可以清晰地看到机器学习方法的优势:

\begin{enumerate}
    \item \textbf{有监督 CNN vs 传统方法}:CNN 的 Hit@10 达到 82.55\%,比最佳传统方法 DTW(70.45\%)\textbf{提升 12.10 个百分点}。这说明即使是简单的有监督分类网络,其学习到的特征表示也显著优于手工设计的 MFCC 特征。

    \item \textbf{对比学习的额外收益}:Contrastive 方法(87.80\%)比普通 CNN(82.55\%)\textbf{再提升 5.25pp}。对比学习目标直接优化嵌入空间的结构,使得同类样本聚集、异类样本分离,更适合检索任务。

    \item \textbf{预训练大模型的巨大优势}:CLAP(99.50\%)比 CNN(82.55\%)\textbf{提升近 17pp},比传统 DTW \textbf{提升 29pp}。这体现了大规模预训练数据和更强模型架构的价值。
\end{enumerate}

\subsubsection{为什么分类模型能用于检索}

分类模型能够用于检索任务的本质原因是:\textbf{分类任务迫使模型学习具有判别性的特征表示}。

\begin{itemize}
    \item 为了正确分类 50 个类别,模型必须学会区分不同类别声音的本质特征
    \item 这些判别性特征自然地使得同类样本在嵌入空间中接近,异类样本远离
    \item 因此,分类模型的中间层特征可以直接用于基于相似度的检索
\end{itemize}

\subsubsection{对比学习 vs 分类学习}

对比学习(Contrastive)比普通分类学习(CNN)更适合检索任务,原因包括:

\begin{itemize}
    \item \textbf{目标函数更匹配}:对比学习直接优化样本间的相似度关系,而分类学习只优化决策边界
    \item \textbf{嵌入空间更均匀}:对比学习产生的嵌入分布更均匀,避免了分类模型可能出现的"特征塌缩"问题
    \item \textbf{对困难样本更鲁棒}:对比学习的负样本挖掘机制使模型更关注难区分的样本对
\end{itemize}

% =========================================================
\section{与大模型的系统对比}

根据作业要求,我们需要将自训练的模型与大模型进行对比。本节系统比较不同规模模型在分类任务上的表现。

\subsection{模型规模与性能对比}

\begin{table}[H]
\centering
\caption{不同规模模型的分类性能系统对比}
\label{tab:model_scale_comparison}
\begin{tabular}{@{}llccc@{}}
\toprule
\textbf{模型} & \textbf{类型} & \textbf{预训练数据} & \textbf{分类准确率 (\%)} & \textbf{检索 Hit@10 (\%)} \\
\midrule
\multicolumn{5}{c}{\textit{小规模模型(从头训练或ImageNet预训练)}} \\
\midrule
ResNet18 (Mel) & CNN & ImageNet & 81.75 & -- \\
ResNet18 (MFCC) & CNN & ImageNet & 70.75 & -- \\
CNN (5层) & CNN & 无 & -- & 82.55 \\
\midrule
\multicolumn{5}{c}{\textit{中规模模型(AudioSet预训练)}} \\
\midrule
CNN14 (PANNs) & CNN & AudioSet (2M) & 92.75 & -- \\
\midrule
\multicolumn{5}{c}{\textit{大规模预训练模型}} \\
\midrule
BEATs (Iter3+) & Transformer & AudioSet (2M) & \textbf{96.50} & 99.10 \\
CLAP (Zero-shot) & 多模态 & AudioSet+Text & 93.90 & -- \\
CLAP (Linear Probe) & 多模态 & AudioSet+Text & 96.50 & 99.50 \\
CLAP (Ultimate) & 多模态 & AudioSet+Text & \textbf{98.00+} & \textbf{99.50} \\
\bottomrule
\end{tabular}
\end{table}

\subsection{关键发现}

\subsubsection{预训练规模的重要性}

\begin{enumerate}
    \item \textbf{从头训练 vs 预训练}:ResNet18 在 ImageNet 上预训练后迁移到音频任务(81.75\%),虽然 ImageNet 是图像数据集,但预训练仍然提供了有用的底层特征提取能力。

    \item \textbf{领域内预训练的优势}:CNN14 在 AudioSet(200万音频样本)上预训练,准确率达到 92.75\%,比 ImageNet 预训练的 ResNet18 \textbf{高出 11pp}。这说明领域内预训练数据的重要性。

    \item \textbf{大模型的显著优势}:BEATs(96.50\%)和 CLAP(98\%+)进一步提升性能,体现了更大模型容量和更先进架构(Transformer、多模态对齐)的价值。
\end{enumerate}

\subsubsection{Zero-shot 能力的惊人表现}

CLAP 的 Zero-shot 分类(无需在 ESC-50 上训练)达到 \textbf{93.90\%},这一结果有重要意义:

\begin{itemize}
    \item \textbf{超越从头训练的小模型}:Zero-shot CLAP(93.90\%)显著超过从头训练的 ResNet18(81.75\%),提升超过 12pp
    \item \textbf{无需标注数据}:Zero-shot 方式完全不需要 ESC-50 的训练数据,仅依靠预训练知识和文本描述
    \item \textbf{实际应用价值}:对于新领域或标注数据稀缺的场景,大模型的 Zero-shot 能力具有巨大的实际价值
\end{itemize}

\subsubsection{架构的影响:CNN vs Transformer}

\begin{table}[H]
\centering
\caption{相同优化策略下 CNN 与 Transformer 架构对比}
\begin{tabular}{@{}lccc@{}}
\toprule
\textbf{模型} & \textbf{架构} & \textbf{增强策略} & \textbf{准确率 (\%)} \\
\midrule
CNN14 & CNN & SpecAugment + Mixup + 解冻 & 92.75 \\
BEATs & Transformer & SpecAugment + Mixup + 解冻 & 96.50 \\
\midrule
\multicolumn{3}{r}{\textbf{Transformer 相对提升}} & \textbf{+3.75pp} \\
\bottomrule
\end{tabular}
\end{table}

在相同的训练策略下,BEATs(Transformer)比 CNN14(CNN)高 3.75pp,原因包括:
\begin{itemize}
    \item \textbf{全局建模能力}:Transformer 的自注意力机制能够捕获长距离依赖,更好地建模音频的全局结构
    \item \textbf{声学标记器}:BEATs 使用离散声学标记进行自监督预训练,学习到更丰富的语义表示
    \item \textbf{数据效率}:Transformer 架构在小数据集(ESC-50 仅 2000 样本)上展现出更强的泛化能力
\end{itemize}

\subsection{小模型 vs 大模型:如何选择}

\begin{table}[H]
\centering
\caption{不同场景下的模型选择建议}
\begin{tabular}{@{}lll@{}}
\toprule
\textbf{场景} & \textbf{推荐模型} & \textbf{理由} \\
\midrule
资源受限 / 边缘部署 & ResNet18 & 参数少、推理快、精度可接受 \\
标准服务器部署 & CNN14 / BEATs & 精度与效率的良好平衡 \\
追求最高精度 & CLAP Ultimate & 98\%+ 准确率,适合精度敏感场景 \\
无标注数据 & CLAP Zero-shot & 无需训练,开箱即用 \\
需要可解释性 & 传统方法 + CNN & 特征可理解,便于调试 \\
\bottomrule
\end{tabular}
\end{table}


\part{总结与感悟}

\section{项目总结}

本项目完整实现了基于 ESC-50 数据集的声音检索与分类系统,主要成果包括:

\begin{enumerate}
    \item \textbf{自实现 DSP 算法}:从零实现了 FFT(Cooley-Tukey 算法)、STFT、MFCC 等核心信号处理算法,并使用 Numba JIT 进行加速优化,达到了与标准库相当的精度(相对误差 $<10^{-10}$)。

    \item \textbf{全面的检索系统}:实现了 13 种检索方法,从传统的池化方法(M1-M4)、DTW(M5)、BoAW(M6-M7),到深度学习方法(CLAP、BEATs),最终 CLAP 达到 99.50\% Hit@10,显著超越 DTW 基线的 70.45\%。

    \item \textbf{多模型分类对比}:系统比较了 ResNet18、CNN14、BEATs、CLAP 等模型,通过 SpecAugment、Mixup、延迟解冻等策略,BEATs 达到 96.50\%,CLAP Ultimate Optimization 达到 98\%+。

    \item \textbf{超参数敏感性分析}:对帧长/帧移、特征类型、CMVN 策略等进行了全面的消融实验,发现 n\_fft=2048, hop\_length=512 为最优配置,全局 CMVN 带来 +4.55pp 提升。
\end{enumerate}

\section{技术收获}

\subsection{信号处理层面}
\begin{itemize}
    \item 深入理解了 FFT 的 Cooley-Tukey 蝶形运算原理
    \item 掌握了 STFT 的时频分析方法与窗函数选择
    \item 理解了 Mel 滤波器组的设计原理与听觉感知基础
    \item 学会了 CMVN、Delta 特征等实用技术
\end{itemize}

\subsection{深度学习层面}
\begin{itemize}
    \item 掌握了预训练模型的迁移学习策略(冻结、解冻、Adapter)
    \item 理解了 SpecAugment、Mixup 等数据增强技术的原理与应用
    \item 学会了 Label Smoothing、TTA 等正则化与推理优化技术
    \item 体会到预训练大模型(CLAP、BEATs)的强大泛化能力
\end{itemize}

\section{团队分工}

\begin{table}[H]
\centering
\begin{tabular}{@{}ll@{}}
\toprule
\textbf{成员} & \textbf{主要贡献} \\
\midrule
刘嘉俊 & 项目统筹、CLAP 微调实验、报告整合 \\
孙浩翔 & 检索系统实现、DTW/BoAW 方法、效率优化 \\
田原 & BEATs 分类实验、消融实验设计 \\
叶栩言 & CNN14 实验、数据增强策略 \\
林梓杰 & DSP 核心算法实现、特征工程 \\
\midrule
\textit{(注:以上分工为虚构示例,请替换为实际分工)} & \\
\bottomrule
\end{tabular}
\caption{团队分工}
\end{table}

\section{遇到的挑战与解决}

\begin{enumerate}
    \item \textbf{FFT 精度问题}:初期实现的 FFT 存在数值不稳定,通过引入 Numba JIT 和优化蝶形运算顺序解决。

    \item \textbf{预训练模型过拟合}:直接微调 BEATs/CLAP 容易过拟合 ESC-50 小数据集,通过冻结编码器 + 延迟解冻策略解决。

    \item \textbf{DTW 效率瓶颈}:原始 DTW 的 $O(N^2)$ 复杂度导致检索速度慢,通过 Sakoe-Chiba 带状约束 + Numba 并行化实现 10$\times$ 加速。

    \item \textbf{CMVN 策略选择}:逐条 CMVN 导致灾难性失败(-51.80pp),发现全局 CMVN 是正确做法。
\end{enumerate}

% =========================================================
\addcontentsline{toc}{section}{参考文献}

\begin{thebibliography}{9}
\bibitem{esc50}
Piczak, K. J. (2015). \textit{ESC: Dataset for Environmental Sound Classification}.

\bibitem{clap}
Elizalde, B., et al. (2023). \textit{CLAP: Learning Audio Concepts from Natural Language Supervision}.

\bibitem{beats}
Chen, S., et al. (2023). \textit{BEATs: Audio Pre-Training with Acoustic Tokenizers}.

\bibitem{dtw}
Sakoe, H., \& Chiba, S. (1978). \textit{Dynamic Programming Algorithm Optimization for Spoken Word Recognition}.

\bibitem{mfcc}
Davis, S., \& Mermelstein, P. (1980). \textit{Comparison of Parametric Representations for Monosyllabic Word Recognition}.
\end{thebibliography}


\end{document}
